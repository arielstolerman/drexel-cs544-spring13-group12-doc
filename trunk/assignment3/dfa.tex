\section{DFA}
\label{sec:dfa}

\begin{figure}[h]
  \centering
  \includegraphics[width=4.95in]{figures/dfa.pdf}
  \caption{DFA illustration for the RSHC protocol.}
  \label{fig:dfa:dfa}
\end{figure}

An illustration of a DFA for RSHC is shown in \figr{fig:dfa:dfa}. Most of the states consist of handshake and initialization, the synchronous phase of the communication. After the main communication is in progress, the asynchronous phase, there are only two states: the client sends an action and awaits a confirmation (bottom right state), or simply idle (bottom left state), in which the client may receive updates from the server on non-client-invoked actions. Note that when the client awaits a confirmation, it can continue sending actions and receiving confirmations. Only when the last confirmation / denial is sent from the server, the state switches back to idle.

We chose to use only two states for the asynchronous phase of the protocol for simplicity. However, we can pair a DFA state to each subset of legal messages, which means a state for every configuration of the house: the states of all monitored devices. This approach would yield a number of states exponential in the number of devices, and a correspondence to the states of the application. Since the two-states model is sufficient for representation, it is chosen for the illustration of the protocol.
