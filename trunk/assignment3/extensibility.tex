\section{Extensibility}
\label{sec:extend}

Extensibility in RSHC is enabled by the initial version handshake between the client and server, as detailed in \sect{sec:pdus:pdu:hs}:
\begin{itemize}
\end{itemize}

Additional built-in device types can be added to subsequent versions of the protocol by simply adding additional built-in types to built-in array in the INIT state.

While there are several device types built into the RSHC protocol, it is expected that a subsequent version of the protocol should support device types not built in to the protocol.  New device types would be supported by passing a device type description and a list of possible device type commands during the INIT state.  With this information the client can manipulate and query a new device just as easily as a built in device.  For example, {\tt c5} completely specifies a custom device type: begins with the number of {\tt c5} devices (just like for preset devices), but follows device description, number of states and their description, number of actions and their encodings, and finally actual instances information -- device names and current state. Note that we leave action encoding format for future development; however we note that it should include parameter information (number, size and order) and states at which the action is legal (the {\tt action enc. size} preceding the action encoding should manage delineation).

\begin{verbatim}
[0x03 : 1]
[n0=#type 0 devices: 1][name0: 16][state0: 1]...[name n0: 16][state n0: 1]
...
[n4=#type 1 devices: 1][name0: 16][state0: 1]...[name n4: 16][state n4: 1]
[c5=#type 1 devices: 1]
    // device description
    [device type description: 16]
    // # device states and their description
    [m=#state count: 1]
    [state0 desc.: 16][state1 desc.: 16]...[state m desc.: 16]
    // # device actions and their encoding
    [a=#action count: 1]
    [A0=action0 enc. size: 1][action0 enc.: A0]
    ...
    [Aa=action0 enc. size: 1][action0 enc.: Aa]
    // finally, name of instances of the device and their current state
    [name0: 16][state0: 1]...[name c5: 16][state c5: 1]
\end{verbatim}

Since it is assumed the transport layer is reliable and connection oriented, new message types can be added by defining them in a subsequent version of the protocol.  No assumptions are made about the DFA that must be carried over to a future version, therefore adding or modifying states is as simple as defining them.  Of course, backward compatibility may be a goal of a future version.  In that case it is recommended that the only way to enter a new state is with a new message type.


