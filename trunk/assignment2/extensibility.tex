\section{Extensibility}
\label{sec:extend}

Extensibility in RSHC is enabled by the initial version handshake between the client and server.

Additional built-in device types can be added to subsequent versions of the protocol by simply adding additional built-in types to built-in array in the INIT state.  

While there are several device types built into the RSHC protocol, it is expected that a subsequent version of the protocol should support device types not built in to the protocol.  New device types would be supported by passing a device type description and a list of possible device type commands during the INIT state.  With this information the client can manipulate and query a new device just as easily as a built in device.  For example, $c5$ completely specifies a custom device type, the number of devices of the custom type, their description, and their current state.

\begin{verbatim}
[0x03 : 1]
[n0=#type 0 devices: 1][name0: 16][state0: 1]...[name n0: 16][state n0: 1]
...
[n4=#type 1 devices: 1][name0: 16][state0: 1]...[name n4: 16][state n4: 1]
[c5=#type 1 devices: 1][device type description: 16][state_count: 1]
[state s0: 16][state s1: 16]...[state sm: 16]
[name n0: 16][state s0: 1]...[name n4: 16][state s4: 1]
\end{verbatim}

Since it is assumed the transport layer is reliable and connection oriented, new message types can be added by defining them in a subsequent version of the protocol.  No assumptions are made about the DFA that must be carried over to a future version, therefore adding or modifying states is as simple as defining them.  Of course, backward compatibility may be a goal of a future version.  In that case it is recommended that the only way to enter a new state is with a new message type.
.  



